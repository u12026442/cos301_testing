
\subsection{Spaces A}

\subsubsection{Maintainability}
There is a lack of the use of development standards such as:
\begin{itemize}
	\item structured programming
		\begin{itemize}
			\item app.js: Line 81-82: Commented out code for unit testing instead of using a unit testing flag that could be set to true when you need to test the unit.
			\item routes/index.js: Line 110-115: Commented out code used for debugging instead of checking if the development flag is set for the system as a whole.
			\item routes/index.js: Line 102-104: Different indentations in code.
		\end{itemize}
	\item recognizable nomenclature
routes/index.js: Line 104 vs 109: "registerUser" is good recognizable nomenclature, where as in line 109 "submitRU" (refering to submit remove user) are very confusing. Same in line 93,78,67,39
	\item standards for the user interfaces
	\begin{itemize}
		\item User interfaces very unorganized in the views folder. No clear names to know what the interface refer to.
		\item views/registerUser.hbs: duplicate <head> tags and <head> tags in the body, in the final file because it will be included as the body of the layout. \\
		No CSS file included for styling. \\
		<title> tag specified again instead of sending the title over in the JSON object.
		\item views/layout.hbs: Hard coded menu (Lines 17-24), makes it difficult to find where the menu items are stored. Should be stored in the database with urls to the routes.
	\end{itemize}
\end{itemize}

Programs have not been parameterized under necessary conditions to promote reusuability:
\begin{itemize}
	\item views/layout.hbs: Hard coded menu (Lines 17-24), makes it difficult to reuse code for other menu items.
\end{itemize}

\subsubsection{Scalability and Performance}

No provision are made for making modules 

\subsubsection{Reliability and Availability}

\subsubsection{Security}

No trouble have been made to check if a user is logged in before adding a space, adding/removing users and admins. eg. routes/index.js: Lines 93-101

\subsubsection{Auditability}

Nothing are being written to log files. Also nothing are being sent to the database audit log.

\subsubsection{Testability}

Testability are made difficult because all the tests are commented lines in the app. (eg. app.js: Line 81-82) Modules like mocha, supertest and should could be used to test the unit without running the actual program.

\subsubsection{Usability}

No able to test because of the lack of an user interface that works. Menu are easy to use and in the same place every time though.

\subsubsection{Deployability}

Very easy to deploy as an app. (Just use "npm start"). No install script for Node and MongoDB so it should be installed beforehand. There could also be a setting.json file made to setup the site name and database path and name for easier deployment.