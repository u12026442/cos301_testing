
\subsection{Spaces A}

\subsubsection{Maintainability}
There is a lack of the use of development standards such as:
\begin{itemize}
	\item structured programming
		\begin{itemize}
			\item index.js: No good commented code for descriptions of functions.
			\item index.js: Line 310-315: Commented out code used for debugging instead of checking if the development flag is set for the system as a whole.
			\item routes/index.js: Line 302-304: Different indentations in code.
		\end{itemize}
	\item recognizable nomenclature
index.js: Line 111 vs 130: "submitNotify" is good recognizable nomenclature, where as in line 109 "submitCS" (refering to submit create space) are very confusing. Same in line 267,278,293,159
\end{itemize}

Programs have not been parameterized under necessary conditions to promote reusuability:
\begin{itemize}
	\item index.js: Line 141-150: Mail Username and password are hard-coded. No settings file for the email settings.
	
\end{itemize}

\subsubsection{Security}

No trouble have been made to check if a user is logged in before adding a space, adding/removing users and admins. eg. routes/index.js: Lines 93-101

\subsubsection{Auditability}

Nothing are being written to log files. Also nothing are being sent to the database audit log.

\subsubsection{Testability}

Testability are made difficult because all the tests are commented lines in the app. (eg. index.js: Line 97-107) Modules like mocha, supertest and should could be used to test the unit without running the actual program.

\subsubsection{Deployability}

Very easy to deploy as an app. (Just use "npm install buzz-spaces"). No install script for Node and MongoDB so it should be installed beforehand. There could also be a setting.json file made to setup the app name, database path and DB name and mail server settings for easier deployment.